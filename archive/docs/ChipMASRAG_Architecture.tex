\begin{figure*}[htbp]
    \centering
    \resizebox{0.98\textwidth}{!}{%
    \begin{tikzpicture}[
        % 定义样式
        coordinator/.style={rectangle, rounded corners=5pt, minimum width=3.5cm, minimum height=1.3cm, 
                            text centered, draw=black, fill=blue!25, thick, font=\small},
        agent/.style={rectangle, rounded corners=5pt, minimum width=2.6cm, minimum height=1.9cm, 
                      text centered, draw=black, fill=green!25, thick, font=\scriptsize, align=center},
        execution/.style={rectangle, rounded corners=5pt, minimum width=2.2cm, minimum height=1cm, 
                          text centered, draw=black, fill=orange!25, thick, font=\small},
        knowledge/.style={cylinder, shape border rotate=90, aspect=0.2, minimum width=2cm, 
                          minimum height=1.6cm, draw=black, fill=yellow!25, thick, font=\tiny, align=center},
        rag/.style={rectangle, rounded corners=3pt, minimum width=1.8cm, minimum height=0.7cm, 
                    text centered, draw=black, fill=purple!20, font=\tiny},
        arrow/.style={thick, ->, >=stealth},
        label/.style={font=\tiny, fill=white, text opacity=0.9, inner sep=1pt}
    ]
    
    % ========== 第一层:协调者层 ==========
    % 输入
    \node[font=\small] (input) at (0, 9.5) {\textbf{设计输入 $D$}};
    
    % 协调者智能体
    \node[coordinator] (coordinator) at (0, 8.2) {\textbf{协调者智能体 $\mathcal{A}_0$}};
    
    % RAG检索模块 - 水平排列,增加间距
    \node[rag] (rag_coarse) at (-4.2, 7) {粗粒度\\检索};
    \node[rag] (rag_fine) at (-1.4, 7) {细粒度\\检索};
    \node[rag] (rag_semantic) at (1.4, 7) {语义\\检索};
    \node[rag] (rag_rank) at (4.2, 7) {排序\\Top-K};
    
    % 全局协调模块
    \node[rag, minimum width=3cm, minimum height=0.9cm, fill=blue!30] (global_coord) at (0, 5.8) {\textbf{全局协调模块}};
    
    % 知识库 - 放在右侧,提高位置
    \node[knowledge] (kb) at (6.2, 7.2) {知识库$\mathcal{K}$};
    
    % ========== 第二层:分区智能体层 ==========
    % 分区智能体 - 增加间距
    \node[agent] (agent1) at (-4.5, 3.8) {
        \textbf{分区智能体}\\
        \textbf{$\mathcal{A}_1$}\\
        \rule{2.2cm}{0.4pt}\\
        GAT编码\\
        Actor-Critic\\
        协商网络
    };
    
    \node[agent] (agent2) at (-1.5, 3.8) {
        \textbf{分区智能体}\\
        \textbf{$\mathcal{A}_2$}\\
        \rule{2.2cm}{0.4pt}\\
        GAT编码\\
        Actor-Critic\\
        协商网络
    };
    
    \node[agent] (agent3) at (1.5, 3.8) {
        \textbf{分区智能体}\\
        \textbf{$\mathcal{A}_3$}\\
        \rule{2.2cm}{0.4pt}\\
        GAT编码\\
        Actor-Critic\\
        协商网络
    };
    
    \node[agent] (agent4) at (4.5, 3.8) {
        \textbf{分区智能体}\\
        \textbf{$\mathcal{A}_4$}\\
        \rule{2.2cm}{0.4pt}\\
        GAT编码\\
        Actor-Critic\\
        协商网络
    };
    
    % ========== 第三层:执行层 ==========
    \node[execution] (layout_gen) at (-3, 1.8) {布局生成};
    \node[execution] (evaluate) at (0, 1.8) {质量评估};
    \node[execution] (feedback) at (3, 1.8) {知识库\\更新};
    
    % 输出
    \node[font=\small] (output) at (3, 0.4) {\textbf{布局解 $Y^*$}};
    
    % ========== 连接线:输入到协调者 ==========
    \draw[arrow] (input) -- (coordinator);
    
    % ========== 连接线:协调者到RAG检索流程 ==========
    % 从协调者底部边缘出发,避免穿过文字
    \draw[arrow] (coordinator.south west) to[out=-90, in=135] ([xshift=-0.2cm]rag_coarse.north west);
    \draw[arrow] (rag_coarse.east) -- (rag_fine.west);
    \draw[arrow] (rag_fine.east) -- (rag_semantic.west);
    \draw[arrow] (rag_semantic.east) -- (rag_rank.west);
    % 从排序Top-K底部边缘出发,避免穿过全局协调模块文字
    \draw[arrow] (rag_rank.south east) to[out=-90, in=45] ([xshift=0.2cm]global_coord.north east);
    
    % ========== 连接线:知识库与RAG检索(使用更长的弯曲路径,避免覆盖文字) ==========
    % 从知识库左侧边缘出发,避免穿过知识库文字
    \draw[arrow] (kb.west) to[out=180, in=30] ([xshift=-0.4cm, yshift=0.1cm]rag_coarse.east);
    \node[above of=rag_coarse, yshift=-0.4cm, font=\tiny, fill=white, text opacity=0.95, inner sep=2pt, rounded corners=2pt] (retrieve_label) {检索};
    % 从排序Top-K右侧边缘出发,避免穿过文字
    \draw[arrow] (rag_rank.east) to[out=0, in=150] ([xshift=0.4cm, yshift=-0.1cm]kb.west);
    \node[above of=rag_rank, yshift=-0.4cm, font=\tiny, fill=white, text opacity=0.95, inner sep=2pt, rounded corners=2pt] (update_label) {更新};
    
    % ========== 连接线:全局协调模块到分区智能体(使用更长的弯曲路径) ==========
    \draw[arrow, dashed, blue, line width=1.3pt] (global_coord.south west) to[out=-150, in=90] ([xshift=0.2cm]agent1.north) node[midway, left, font=\tiny, fill=white, text opacity=0.95, inner sep=2pt, rounded corners=2pt, xshift=-0.3cm] {RAG结果};
    \draw[arrow, dashed, blue, line width=1.3pt] (global_coord.south) to[out=-90, in=90] (agent2.north);
    \draw[arrow, dashed, blue, line width=1.3pt] (global_coord.south) to[out=-90, in=90] (agent3.north);
    \draw[arrow, dashed, blue, line width=1.3pt] (global_coord.south east) to[out=-30, in=90] ([xshift=-0.2cm]agent4.north) node[midway, right, font=\tiny, fill=white, text opacity=0.95, inner sep=2pt, rounded corners=2pt, xshift=0.3cm] {RAG结果};
    
    % ========== 连接线:智能体间协商(标签放在连接线正上方中间) ==========
    \draw[arrow, <->, red, line width=1.1pt] (agent1.east) -- (agent2.west) node[midway, above, font=\tiny, fill=white, text opacity=0.95, inner sep=2pt, rounded corners=2pt, yshift=0.15cm] {边界协商};
    \draw[arrow, <->, red, line width=1.1pt] (agent2.east) -- (agent3.west) node[midway, above, font=\tiny, fill=white, text opacity=0.95, inner sep=2pt, rounded corners=2pt, yshift=0.15cm] {边界协商};
    \draw[arrow, <->, red, line width=1.1pt] (agent3.east) -- (agent4.west) node[midway, above, font=\tiny, fill=white, text opacity=0.95, inner sep=2pt, rounded corners=2pt, yshift=0.15cm] {边界协商};
    
    % ========== 连接线:智能体到执行层(使用弯曲路径避免重叠和覆盖文字) ==========
    % 从智能体底部边缘出发,避免穿过内部文字
    \draw[arrow] (agent1.south west) to[out=-90, in=135] ([xshift=-0.3cm]layout_gen.north west);
    \draw[arrow] (agent2.south) to[out=-90, in=90] ([xshift=-0.1cm]layout_gen.north);
    \draw[arrow] (agent3.south) to[out=-90, in=90] ([xshift=0.1cm]layout_gen.north);
    \draw[arrow] (agent4.south east) to[out=-90, in=45] ([xshift=0.3cm]layout_gen.north east);
    % 执行层内部连接,使用边缘锚点避免覆盖文字
    \draw[arrow] (layout_gen.east) -- (evaluate.west);
    \draw[arrow] (evaluate.east) -- (feedback.west);
    % 从知识库更新顶部边缘出发,避免穿过文字
    \draw[arrow] (feedback.north east) to[out=45, in=-135] ([xshift=0.3cm, yshift=-0.2cm]kb.south);
    
    % ========== 连接线:执行层反馈到智能体(标签放在反馈箭头中间位置,靠近连接线) ==========
    % 反馈到A1:从质量评估左侧出发,向左上方弯曲,然后直接指向A1的底部
    \draw[arrow, dashed, green!70!black, line width=1.3pt] (evaluate.west) to[out=180, in=0] ([xshift=-1.8cm, yshift=1.2cm]evaluate.center) to[out=180, in=-135] (agent1.south west) node[pos=0.5, above, font=\tiny, fill=white, text opacity=0.95, inner sep=2pt, rounded corners=2pt, yshift=0.1cm] {质量反馈};
    % 反馈到A4:从质量评估右侧出发,向右上方弯曲,然后直接指向A4的底部
    \draw[arrow, dashed, green!70!black, line width=1.3pt] (evaluate.east) to[out=0, in=180] ([xshift=1.8cm, yshift=1.2cm]evaluate.center) to[out=0, in=-45] (agent4.south east) node[pos=0.5, above, font=\tiny, fill=white, text opacity=0.95, inner sep=2pt, rounded corners=2pt, yshift=0.1cm] {质量反馈};
    
    % ========== 连接线:输出(使用边缘锚点避免覆盖文字) ==========
    \draw[arrow] (evaluate.south east) to[out=-90, in=90] ([xshift=0.1cm]output.north);
    
    % ========== 标注层(放在最左侧边缘,确保不覆盖任何图案) ==========
    \node[font=\bfseries\small, anchor=east] (layer1) at (-6.5, 8.2) {协调者层};
    \node[font=\bfseries\small, anchor=east] (layer2) at (-6.5, 3.8) {分区智能体层};
    \node[font=\bfseries\small, anchor=east] (layer3) at (-6.5, 1.8) {执行层};
    
    % ========== 图例(调整位置) ==========
    \node[draw, rectangle, fill=white, minimum width=2.8cm, minimum height=1.8cm, 
          anchor=north east, font=\tiny, align=left, text width=2.6cm] (legend) at (7, 0.6) {
        \textbf{图例:}\\
        \textcolor{blue}{--- RAG检索流}\\
        \textcolor{red}{--- 边界协商}\\
        \textcolor{green!70!black}{--- 质量反馈}\\
        \textcolor{black}{--- 数据流}
    };
    
    \end{tikzpicture}%
    }%
    \caption{ChipMASRAG系统架构图。架构包含三层:(1)协调者层 - RAG统一检索(粗粒度→细粒度→语义检索→排序Top-K)、全局协调、知识库管理;(2)分区智能体层 - 4个分区智能体,每个包含GAT状态编码、Actor-Critic策略网络、协商网络,智能体间通过知识驱动的边界协商协议协作;(3)执行层 - 布局生成、质量评估、知识库更新。蓝色虚线表示RAG检索结果从协调者广播到所有智能体,红色双向箭头表示智能体间的边界协商,绿色虚线表示质量反馈从执行层返回到智能体。}
    \label{fig:architecture}
    \end{figure*}
    
    